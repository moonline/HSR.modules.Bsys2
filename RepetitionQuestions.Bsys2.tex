%Pakete;
%A4, Report, 12pt
\documentclass[ngerman,a4paper,12pt]{scrreprt}
\usepackage[a4paper, right=20mm, left=20mm,top=20mm, bottom=30mm, marginparsep=5mm, marginparwidth=5mm, headheight=7mm, headsep=15mm,footskip=15mm]{geometry}

%Papierausrichtungen
\usepackage{pdflscape}
\usepackage{lscape}

%Deutsche Umlaute, Schriftart, Deutsche Bezeichnungen
\usepackage[utf8]{inputenc}
\usepackage[T1]{fontenc}
\usepackage[ngerman]{babel}

%quellcode
\usepackage{listings}

%tabellen
\usepackage{tabularx}

%listen und aufzählungen
\usepackage{paralist}

%farben
\usepackage[svgnames,table,hyperref]{xcolor}

%symbole
\usepackage{latexsym,textcomp}

%font
\usepackage{helvet}
\renewcommand{\familydefault}{\sfdefault}

%Abkürzungsverzeichnisse
\usepackage[printonlyused]{acronym}

%Bilder
\usepackage{graphicx} %Bilder
\usepackage{float}	  %"Floating" Objects, Bilder, Tabellen...
\usepackage[space]{grffile} %Leerzechen Problem bei includegraphics
\usepackage{wallpaper} %Seitenhintergrund setzen
\usepackage{transparent} %Transparenz

%for
\usepackage{forloop}
\usepackage{ifthen}

%Dokumenteigenschaften
\title{Repetitionsfragen Bsys2}
\author{Tobias Blaser}
\date{\today{}, Rapperswil}


%Kopf- /Fusszeile
\usepackage{fancyhdr}
\usepackage{lastpage}

\pagestyle{fancy}
	\fancyhf{} %alle Kopf- und Fußzeilenfelder bereinigen
	\renewcommand{\headrulewidth}{0pt} %obere Trennlinie
	\fancyfoot[L]{Seite \thepage/\pageref{LastPage}} %Fusszeile mitte
	\fancyfoot[R]{\today{}} %Fusszeile rechts
	\renewcommand{\footrulewidth}{0.4pt} %untere Trennlinie

%Kopf-/ Fusszeile auf chapter page
\fancypagestyle{plain} {
	\fancyhf{} %alle Kopf- und Fußzeilenfelder bereinigen
	\renewcommand{\headrulewidth}{0pt} %obere Trennlinie
	\fancyfoot[L]{Seite \thepage/\pageref{LastPage}} %Fusszeile mitte
	\fancyfoot[R]{\today{}} %Fusszeile rechts
	\renewcommand{\footrulewidth}{0.4pt} %untere Trennlinie
}

\usepackage{changepage}

% Abkürzungen für Kapitel, Titel und Listen
\input{commands/shortcutsListAndChapter}
\input{commands/TextStructuringBoxes}

%links, verlinktes Inhaltsverzeichnis, PDF Inhaltsverzeichnis
\usepackage[bookmarks=true,
bookmarksopen=true,
bookmarksnumbered=true,
breaklinks=true,
colorlinks=true,
linkcolor=black,
anchorcolor=black,
citecolor=black,
filecolor=black,
menucolor=black,
pagecolor=black,
urlcolor=black
]{hyperref} % Paket muss unbedingt als letzes eingebunden werden!

\usepackage{graphicx}
\begin{document}

% Inhaltsverzeichnis
\tableofcontents
\clearpage

\ch{Repetition Bsys2}
\ol
	\li Nennen Sie ein Paar Gründe für Sicherheitsprobleme bei Software.
	\li Erklären Sie die vier Sicherheitsgefährdungen.
	\li Was braucht ein Betriebsystem, um Sicherheit zu gewährleisten?
	\li Aus welchen zwei Unteraufgaben setzen sich Rechtezuordnung und Rechteüberprüfung jeweils zusammen.
	\li Aus welchen drei Blöcken setzt sich ein Schutzsystem zusammen?
	\li Erklären Sie das Schutzdomänenkonzept
	\li Erkläre Sie die Schutzdomänenmatrix.
	\li Erklären Sie den Unterschied zwischen Schutzstrategie und Schutzmechanismus.
	\li Was sind Verwaltungsrechte?
	\li Erklären Sie die drei Ausführungen des Kopierrechtes.
	\li Erklären Sie das Eignerkonzept.
	\li Erklären Sie das Kontollrecht.
	\li Zeichnen Sie in der Schutzdomänenmatrix ein, welche Rechte ein Benutzer beim Eignerkonzept und beim Kontrollkonzept verändern kann.
	\li Wie wird eine Schutzmatrix implementiert? Nennen Sie zu jeder Variante vor- und Nachteile. Berücksichtigen Sie dabei auch Löschvorgänge.
	\li Erklären Sie den Unterschied zwischen Benutzerbestimmter und Systembestimmter Zugriffskontrolle.
	\li Was ist das Prinzip der minimalen Privilegien?
	\li Erklären Sie das Sicherheitskonzept von Unix? Wie sind die Dateirechte aufgebaut?
	\li Was ist die umask?
	\li Was ist Benutzersbstitution?
\olS


\end{document}
