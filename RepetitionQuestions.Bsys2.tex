%Pakete;
%A4, Report, 12pt
\documentclass[ngerman,a4paper,12pt]{scrreprt}
\usepackage[a4paper, right=20mm, left=20mm,top=20mm, bottom=30mm, marginparsep=5mm, marginparwidth=5mm, headheight=7mm, headsep=15mm,footskip=15mm]{geometry}

%Papierausrichtungen
\usepackage{pdflscape}
\usepackage{lscape}

%Deutsche Umlaute, Schriftart, Deutsche Bezeichnungen
\usepackage[utf8]{inputenc}
\usepackage[T1]{fontenc}
\usepackage[ngerman]{babel}

%quellcode
\usepackage{listings}

%tabellen
\usepackage{tabularx}

%listen und aufzählungen
\usepackage{paralist}

%farben
\usepackage[svgnames,table,hyperref]{xcolor}

%symbole
\usepackage{latexsym,textcomp}

%font
\usepackage{helvet}
\renewcommand{\familydefault}{\sfdefault}

%Abkürzungsverzeichnisse
\usepackage[printonlyused]{acronym}

%Bilder
\usepackage{graphicx} %Bilder
\usepackage{float}	  %"Floating" Objects, Bilder, Tabellen...
\usepackage[space]{grffile} %Leerzechen Problem bei includegraphics
\usepackage{wallpaper} %Seitenhintergrund setzen
\usepackage{transparent} %Transparenz

%for
\usepackage{forloop}
\usepackage{ifthen}

%Dokumenteigenschaften
\title{Repetitionsfragen Bsys2}
\author{Tobias Blaser}
\date{\today{}, Rapperswil}


%Kopf- /Fusszeile
\usepackage{fancyhdr}
\usepackage{lastpage}

\pagestyle{fancy}
	\fancyhf{} %alle Kopf- und Fußzeilenfelder bereinigen
	\renewcommand{\headrulewidth}{0pt} %obere Trennlinie
	\fancyfoot[L]{Seite \thepage/\pageref{LastPage}} %Fusszeile mitte
	\fancyfoot[R]{\today{}} %Fusszeile rechts
	\renewcommand{\footrulewidth}{0.4pt} %untere Trennlinie

%Kopf-/ Fusszeile auf chapter page
\fancypagestyle{plain} {
	\fancyhf{} %alle Kopf- und Fußzeilenfelder bereinigen
	\renewcommand{\headrulewidth}{0pt} %obere Trennlinie
	\fancyfoot[L]{Seite \thepage/\pageref{LastPage}} %Fusszeile mitte
	\fancyfoot[R]{\today{}} %Fusszeile rechts
	\renewcommand{\footrulewidth}{0.4pt} %untere Trennlinie
}

\usepackage{changepage}

% Abkürzungen für Kapitel, Titel und Listen
\input{commands/shortcutsListAndChapter}
\input{commands/TextStructuringBoxes}

%links, verlinktes Inhaltsverzeichnis, PDF Inhaltsverzeichnis
\usepackage[bookmarks=true,
bookmarksopen=true,
bookmarksnumbered=true,
breaklinks=true,
colorlinks=true,
linkcolor=black,
anchorcolor=black,
citecolor=black,
filecolor=black,
menucolor=black,
pagecolor=black,
urlcolor=black
]{hyperref} % Paket muss unbedingt als letzes eingebunden werden!

\usepackage{graphicx}
\begin{document}

% Inhaltsverzeichnis
\tableofcontents
\clearpage


\ch{Sicherheit}
\ol
	\li Nennen Sie ein Paar Gründe für Sicherheitsprobleme bei Software.
	\li Erklären Sie die vier Sicherheitsgefährdungen.
	\li Was braucht ein Betriebsystem, um Sicherheit zu gewährleisten?
	\li Aus welchen zwei Unteraufgaben setzen sich Rechtezuordnung und Rechteüberprüfung jeweils zusammen.
	\li Aus welchen drei Blöcken setzt sich ein Schutzsystem zusammen?
	\li Erklären Sie das Schutzdomänenkonzept
	\li Erkläre Sie die Schutzdomänenmatrix.
	\li Erklären Sie den Unterschied zwischen Schutzstrategie und Schutzmechanismus.
	\li Was sind Verwaltungsrechte?
	\li Erklären Sie die drei Ausführungen des Kopierrechtes.
	\li Erklären Sie das Eignerkonzept.
	\li Erklären Sie das Kontollrecht.
	\li Zeichnen Sie in der Schutzdomänenmatrix ein, welche Rechte ein Benutzer beim Eignerkonzept und beim Kontrollkonzept verändern kann.
	\li Wie wird eine Schutzmatrix implementiert? Nennen Sie zu jeder Variante vor- und Nachteile. Berücksichtigen Sie dabei auch Löschvorgänge.
	\li Erklären Sie den Unterschied zwischen Benutzerbestimmter und Systembestimmter Zugriffskontrolle.
	\li Was ist das Prinzip der minimalen Privilegien?
	\li Erklären Sie das Bell/La Padula Modell und da Biba Modell.
	\li In welchem Spannungsfeld stehen SIe bei hochsicheren Betriebsystemen? Inwiefern müssen beid der Sicherheit abstriche gemacht werden?
\olS

\se{Unix Sicherheit}
\olR
	\li Erklären Sie das Sicherheitskonzept von Unix? Wie sind die Dateirechte aufgebaut?
	\li Erklären Sie, was unter Unix Domänen sind, welche Objekte es gibt und welche Rechte insgesammt zur Verfügung stehen.
	\li Welche Schutzstrategie wird bei Unix eingesetzt?
	\li Nennen Sie Vor- / Nachteile des Unix Sicherheitssystems.
	\li Was für Rechte können Sie mit setuid / setgid srtzen? Welche Vorteile bietet dieser Mechanismus?
	\li Was ist die umask?
	\li Was ist Benutzersbstitution?
\olS

\se{Windows Sicherheit}
\olR
	\li Erklären Sie das Sicherheitskonzept unter Windows
	\li Was sind unter Windows Domänen, welche Objekte gibt es? Was für Rechte stehen insgesammt zur Verfügung?
	
\olS


\ch{Unix Scripting}
\olR
	\li Warum gibt es unter Unix verschiedene Shells? Nennen Sie die am Meisten verbreiteten.
	\li Erklären Sie, was und Unix Skript ist.
	\li Nennen Sie vier typische Anwendungsgebiete für Skripte.
	\li Erklären Sie, was ``Expandierungen'' bei der Verarbeitung von Skripten sind.
	\li Welchem Zweck dient die Einführungszeile in Skripten?
\olS


\ch{Ein-/Ausgabe}
\olR
	\li Erklären Sie die wichtigsten drei E-/A-Techniken
	\li Vergleichen Sie die drei. Welche bietet in welchem Fall vorteile? Welche ist grundsätzlich nicht anzuwenden und welche lastet den Prozessor am wenigsten aus?
	\li Was ist ein Treiber? Wozu wird er verwendet?
	\li Welche Arten von Ausgabebufferung gibt es?
\olS






\end{document}
