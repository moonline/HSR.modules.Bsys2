%Pakete;
%A4, Report, 12pt
\documentclass[ngerman,a4paper,12pt]{scrreprt}
\usepackage[a4paper, right=20mm, left=20mm,top=20mm, bottom=30mm, marginparsep=5mm, marginparwidth=5mm, headheight=7mm, headsep=15mm,footskip=15mm]{geometry}

%Papierausrichtungen
\usepackage{pdflscape}
\usepackage{lscape}

%Deutsche Umlaute, Schriftart, Deutsche Bezeichnungen
\usepackage[utf8]{inputenc}
\usepackage[T1]{fontenc}
\usepackage[ngerman]{babel}

%quellcode
\usepackage{listings}

%tabellen
\usepackage{tabularx}

%listen und aufzählungen
\usepackage{paralist}

%farben
\usepackage[svgnames,table,hyperref]{xcolor}

%symbole
\usepackage{latexsym,textcomp}

%font
\usepackage{helvet}
\renewcommand{\familydefault}{\sfdefault}

%Abkürzungsverzeichnisse
\usepackage[printonlyused]{acronym}

%Bilder
\usepackage{graphicx} %Bilder
\usepackage{float}	  %"Floating" Objects, Bilder, Tabellen...
\usepackage[space]{grffile} %Leerzechen Problem bei includegraphics
\usepackage{wallpaper} %Seitenhintergrund setzen
\usepackage{transparent} %Transparenz

%for
\usepackage{forloop}
\usepackage{ifthen}

%Dokumenteigenschaften
\title{Repetitionsfragen Bsys2}
\author{Tobias Blaser}
\date{\today{}, Rapperswil}


%Kopf- /Fusszeile
\usepackage{fancyhdr}
\usepackage{lastpage}

\pagestyle{fancy}
	\fancyhf{} %alle Kopf- und Fußzeilenfelder bereinigen
	\renewcommand{\headrulewidth}{0pt} %obere Trennlinie
	\fancyfoot[L]{Seite \thepage/\pageref{LastPage}} %Fusszeile mitte
	\fancyfoot[R]{\today{}} %Fusszeile rechts
	\renewcommand{\footrulewidth}{0.4pt} %untere Trennlinie

%Kopf-/ Fusszeile auf chapter page
\fancypagestyle{plain} {
	\fancyhf{} %alle Kopf- und Fußzeilenfelder bereinigen
	\renewcommand{\headrulewidth}{0pt} %obere Trennlinie
	\fancyfoot[L]{Seite \thepage/\pageref{LastPage}} %Fusszeile mitte
	\fancyfoot[R]{\today{}} %Fusszeile rechts
	\renewcommand{\footrulewidth}{0.4pt} %untere Trennlinie
}

\usepackage{changepage}

% Abkürzungen für Kapitel, Titel und Listen
\input{commands/shortcutsListAndChapter}
\input{commands/TextStructuringBoxes}

%links, verlinktes Inhaltsverzeichnis, PDF Inhaltsverzeichnis
\usepackage[bookmarks=true,
bookmarksopen=true,
bookmarksnumbered=true,
breaklinks=true,
colorlinks=true,
linkcolor=black,
anchorcolor=black,
citecolor=black,
filecolor=black,
menucolor=black,
pagecolor=black,
urlcolor=black
]{hyperref} % Paket muss unbedingt als letzes eingebunden werden!

\usepackage{graphicx}
\begin{document}

% Inhaltsverzeichnis
\tableofcontents
\clearpage


\ch{Sicherheit}
\ol
	\li Nennen Sie ein Paar Gründe für Sicherheitsprobleme bei Software.
	\li Erklären Sie die vier Sicherheitsgefährdungen.
	\li Was braucht ein Betriebsystem, um Sicherheit zu gewährleisten?
	\li Aus welchen zwei Unteraufgaben setzen sich Rechtezuordnung und Rechteüberprüfung jeweils zusammen.
	\li Aus welchen drei Blöcken setzt sich ein Schutzsystem zusammen?
	\li Erklären Sie das Schutzdomänenkonzept
	\li Erkläre Sie die Schutzdomänenmatrix.
	\li Erklären Sie den Unterschied zwischen Schutzstrategie und Schutzmechanismus.
	\li Was sind Verwaltungsrechte?
	\li Erklären Sie die drei Ausführungen des Kopierrechtes.
	\li Erklären Sie das Eignerkonzept.
	\li Erklären Sie das Kontollrecht.
	\li Zeichnen Sie in der Schutzdomänenmatrix ein, welche Rechte ein Benutzer beim Eignerkonzept und beim Kontrollkonzept verändern kann.
	\li Wie wird eine Schutzmatrix implementiert? Nennen Sie zu jeder Variante vor- und Nachteile. Berücksichtigen Sie dabei auch Löschvorgänge.
	\li Erklären Sie den Unterschied zwischen Benutzerbestimmter und Systembestimmter Zugriffskontrolle.
	\li Was ist das Prinzip der minimalen Privilegien?
	\li Erklären Sie das Bell/La Padula Modell und da Biba Modell.
	\li In welchem Spannungsfeld stehen SIe bei hochsicheren Betriebsystemen? Inwiefern müssen beid der Sicherheit abstriche gemacht werden?
\olS

\se{Unix Sicherheit}
\olR
	\li Erklären Sie das Sicherheitskonzept von Unix? Wie sind die Dateirechte aufgebaut?
	\li Erklären Sie, was unter Unix Domänen sind, welche Objekte es gibt und welche Rechte insgesammt zur Verfügung stehen.
	\li Welche Schutzstrategie wird bei Unix eingesetzt?
	\li Nennen Sie Vor- / Nachteile des Unix Sicherheitssystems.
	\li Was für Rechte können Sie mit setuid / setgid srtzen? Welche Vorteile bietet dieser Mechanismus?
	\li Was ist die umask?
	\li Was ist Benutzersbstitution?
\olS

\se{Windows Sicherheit}
\olR
	\li Erklären Sie das Sicherheitskonzept unter Windows
	\li Was sind unter Windows Domänen, welche Objekte gibt es? Was für Rechte stehen insgesammt zur Verfügung?
	
\olS


\ch{Unix Scripting}
\olR
	\li Warum gibt es unter Unix verschiedene Shells? Nennen Sie die am Meisten verbreiteten.
	\li Erklären Sie, was und Unix Skript ist.
	\li Nennen Sie vier typische Anwendungsgebiete für Skripte.
	\li Erklären Sie, was ``Expandierungen'' bei der Verarbeitung von Skripten sind.
	\li Welchem Zweck dient die Einführungszeile in Skripten?
\olS


\ch{Ein-/Ausgabe}
\olR
	\li Erklären Sie die wichtigsten drei E-/A-Techniken
	\li Vergleichen Sie die drei. Welche bietet in welchem Fall vorteile? Welche ist grundsätzlich nicht anzuwenden und welche lastet den Prozessor am wenigsten aus?
	\li Was ist ein Treiber? Wozu wird er verwendet?
	\li Welche Arten von Ausgabebufferung gibt es?
\olS


\ch{X Window System}
\olp{
	\li Warum wurde das X Window System geschaffen?
	\li Welche Vorteile besitzt das X Window System, warum hat es sich bis heute gehalten?
	\li Skizzieren SIe, wie eine rein Textorientierte Benutzerschnittstelle funktioniert
	\li Was ist die Aufgabe des X Window System?
	\li Skizzieren Sie, wie das X Window System aufgebaut ist.
	\li Was ist ein Window Manager? Welche Aufgaben übernimmter er? Wie arbeitet der Window Manager mit dem X Server/Client zusammen?
	\li Was stellen die X Toolkits zur Verfügung?
	\li Was sind X-Desktops?
	\li Was ist das X Protokoll?
	\li Erklären Sie, wie die Nachrichtenpufferung bei X funktioniert. Welche Arten von Nachrichten gibt es?
	\li Was sind X Ressourcen?
	\li Erklären Sie den Aufbau der X Fensterverwaltung. Was ist das Root Window?
	\li Welche Grafikfunktionen stellt der X bereit?
	\li Was ist die Farbtabelle? Wie funktioniert sie? Warum werden nicht direkt RGB Werte verwendet?
	\li Wie funktioniert die Ereignisbehandlung bei X? Welches Window erhält das Ereignis? Was passiert anschliessend mit dem Ereignis? Wie erfahren Elternwindows davon? Was ist die Event Mask?
}


\ch{Windows GUI}
\olp{
	\li Erklären Sie den Unterschied zwischen Programmgesteuerten Abläufen und zwischen Ereignisgesteuerten Abläufen. Ordnen Sie GUI, Stapelverarbeitung/Commandline den beiden Begriffen zu un d erklären Sie den Zusammenhang.
	\li Was sind Ereignismeldungen (windows messages)? Wie sind sie aufgebaut?
	\li Erklären Sie die beiden typischen Thread Typen.
	\li Welche Ereignis Meldungstypen gibt es?
	\li Skizzieren Sie den grundlegenden Aufbau eines Fensters
	\li Wie wird ein Fenster erzeugt und angezeigt?
	\li Wie werden die Meldungen an die Applikation übermittelt? Erklären Sie beide Varianten und nennen Sie vor-/Nachteile und Anwendungsgebiet.
	\li Wie werden Betriebssystemereignisse behandelt und wie Benutzer Thread Ereignisse?
	\li Was ist das virtual Kexboard? Wozu dient es?
	\li Erklären Sie die Fensterhirarchie unter Windows. Was sind Child Window und was owned Window? Was ist die Z-Order?
	\li  Wie erkennt die Fensterprozedur des Elternfensters die Ereignisse des Kindfensters?
	\li  Wie kann das Elternfenster das Aussehen/Verhalten eines Dialogelements konfigurieren?
}


\ch{Speichersystem}
\olp{
	\li Was ist der Primärspeicher und was der Sekundärspeicher? Nennen Sie je einige Eigenschaften und Unterscheidungsmerkmale.
	\li Nennen Sie sechs Speicherprinzipien und erklären Sie jede kurz.
	\li Wie funktioniert ein CAM Speicher? Machen Sie ein einfaches Beispiel mit einigen Rows mit Bit, eine Mask und einem Suchmuster.
	\li Was ist der Lokalitätseffekt?
	\li Wozu dienen Speicherhirarchien? Wie funktionieren Sie?
}
\se{Cache Speicher}
\uli{
	\li Wozu dienen die Cache Speicher?
	\li Zugrffszeit Cache: 1.1ns, Zugriffszeit HS: 10.5ns, 88\% Trefferrate beim Zugriff. Berechnen Sie die mittlere Zugriffszeit.
	\li Erklären Sie das Grundprinzip des Cache Speichers und sein Aufbau. Wie läuft ein lesender Zugriff ab?
	\li Welche Probleme mit dem Cache ergeben sich bei Multi Core Systemen?
	\li Nennen Sie einige Nachteile des Cache Speichers.
	\li Welchen Einfluss hat der Cache Speicher auf die Leistung?
	\li Sie stellen fest, dass beim Kopieren von Daten auf ihre SSD bei Dateien die kleiner als 5MiB die Schreibgeschwindigkeit viel höher ist als bei grösseren Dateien. Woher kommt dieser Effekt und ist dies wirklich so?
	\li Ihr Festplattentreiber arbeitet ohne DMA und ohne Interrupt Signale. Zeichnen Sie den Datenfluss im Bereich des Hauptspeichers, L1, L2, L3 Cache und der CPU. Nutzen die Caches bei dieser Operation etwas? Wenn ja, wie?
}

\se{Heap}
\olp{
	\li Was ist der Heap? Machen Sie eine Schematische Skizze des Speichers, in der Sie sowohl für das Betriebssystem wie für eine Applikation Bereiche für Heap, Stack, Argumente, Daten, etc. einzeichnen.
	\li WIe unterscheidet sich der Heap vom Stack? Nennen Sie typische Daten, die auf dem Stack abgelegt werden und nennen Sie typische Daten, die auf dem Heap abgelegt werden.
	\li Wozu braucht es den Heap? Warum können nicht alle Daten auf dem Stack abgelegt werden?
	\li Erklären Sie wie Sie auf dem Heap Speicher allozieren und wieder freigeben für C, C++, Java. Was passiert, wenn Sie vergessen, den Speicher wieder freizugeben? Warum ist dies bei einem Server tragischer als bei einem Desktop System?
	\li Für welche Systeminterne Aufgaben wird der Heap verwendet?	
	\li Erklären Sie die Grundprinzipien der vier Heaporganisationsformen, die es gibt. Nennen Sie je Vor- und Nachteile und machen Sie je eine Skizze, wie sich der Speicher verändert und wie die Verwaltungsdaten dazu aussehen. Erklären Sie auch die jeweiligen Suchalgorithmen für freien Speicher.
	\li Können SIe in einem Programm die Rekombination des Heap beeinflussen?
	\li Was sind Metadaten der Heap Verwaltung? Wo werden Sie abgelegt? Nennen Sie je Vor- und Nachteile.
	\li Erklären Sie interne und externe Fragmentierung.
	\li Was ist Kompaktierung?
	\li Was sind Master Pointer? Warum braucht es sie?
	\li Welche Probleme mit Echtzeitanwendungen ergeben sich bei einer Heap Verwaltung mit variabler Zuordnungsgrösse?
	\li Skizzieren Sie einen Belegungsplan.
}



\ch{Prozessadressräume}
\olp{
	\li Was sind die .bss, .data und .text Section, die von Compilern generiert werden?	
	\li Was macht das Betriebssystem mit den jeweiligen Sections beim Laden?
	\li Was ist ein Memory Mapped File?
	\li Was ist eine Region, Welche Informationen enthält sie?
}


\ch{Programmübersetzung}
\olp{
	\li Skizzieren Sie, was bei der Programmübersetzung abläuft.
	\li Erklären Sie den Unterscheid zwischen Einschritt-Übersetzung und Mehrweg Übersetzung.
	\li Wozu dient die Zwischensprache bei der Programmübersetzung?
}

\end{document}
